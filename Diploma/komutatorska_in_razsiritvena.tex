

\section{Komutatorska in razširitvena lema}

\subsection{Komutatorska lema}



Recimo, da poznamo zakone v nekaterih podmnožicah grupe $G$, zanima pa nas, kako bi iz njih zgradili zakone za večje podmnožice te grupe. Na to vprašanje odgovarjata komutatorska in razširitvena lema,
ki sta ključni orodji pri obravnavanju zakonov. Začeli bomo z dokazom komutatorske leme, za katero bomo potrebovali naslednjo definicijo.

\begin{definicija}\label{def_netrivialna_potenca}
Naj bo $G$ grupa. Element $g \in  G$ je \emph{periodičen}, če je oblike $g = h^{n}$ za nek element $h \in G$ in naravno število $n \ge  2$. Sicer rečemo, da je $g$ \emph{aperiodičen}.
\end{definicija}

% \begin{definicija}\label{def_netrivialna_potenca}
%     Naj bo $G$ grupa. Element $g \in  G$ je netrivialna potenca, če obstajata $h \in G$ in naravno število $n > 1$, da je $g = h^{n}$.
% \end{definicija}

% \begin{lema}[\cite{Lyndon_Schupp_2015}[str.~8, trditev 2.6]]\label{lem_uporaba_nielsen}
% Vsaka končnogenerirana podgrupa proste grupe je prosta.   
% \end{lema}
% % \begin{skica}
% %     -najprej Nielsenove transformacije
% %     -nato N-okrajšanost
% %     -ideja, da lahko prevedeš vse U na N-okrajšane
% %     -lema, da N-okrajšane ravno pravšnje
% % Ta lema je posebni primer Nielsen-Schreierjevega izreka, ki trdi, da je vsaka podgrupa proste grupe prosta \cite{Lyndon_Schupp_2015}[str.~8, trditev 2.11]. Njen dokaz je podan na straneh 5--7 v istega vira,  

% % \end{skica}

% Ta lema je posebni primer klasičnega Nielsen--Schreierjevega izreka, ki pravi, da je vsaka podgrupa proste grupe prosta. Dokazana je v \cite{Lyndon_Schupp_2015}[5--7] z uporabo Nielsenovih transformacij. 

\begin{lema}
\label{lem_posledica_nielsen_schreier}
Naj bosta besedi $w_1, w_2 \in F_2 = \langle a, b \rangle$. Potem velja natanko ena izmed trditev. \begin{enumerate}
    \item Besedi $w_1$ in $w_2$ komutirata in sta periodični z isto osnovo: Obstaja element $c \in F_2$ ter števili $k_1, k_2 \in \mathbb{Z} \setminus \left\{ 0\right\}$, da je $w_1= c^{k_1}$, $w_2 = c^{k_2}$. 
    \item Podgrupa $\langle w_1, w_2 \rangle \subseteq F_2 = \langle a, b \rangle$ je izomorfna prosti grupi $F_2$.
\end{enumerate}
\end{lema}
\begin{dokaz}
    TODO
\end{dokaz}

Osnovna ideja komutatorske leme je pravzaprav preprosta. Recimo, da imamo besedi $w_1, w_2 \in F_2 \setminus \left\{ 1_{F_2}\right\}  = \langle a,b \rangle$, ki jima pripadata izginjajoči množici $Z(G, w_1)$ ter $Z(G, w_2)$.
oglejmo si komutator $w = [w_1, w_2]$. Če vzamemo par $(g, h) \in Z(G, w_1)$, bo veljalo \begin{equation*}
w(g,h) = [w_1(g,h), w_2(g,h)] = [1_{F_2}, w_2(g,h)] = 1_{F_2}.
\end{equation*}  
  Seveda velja simetrično tudi za pare druge izginjajoče množice. Od tod sledi sklep \begin{equation*}
  Z(G, [w_1, w_2]) \supseteq Z(G, w_1) \cup Z(G, w_2).
  \end{equation*}  
Glavni problem, na katerega lahko naletimo pri tej konstrukciji, je potencialna trivialnost komutatorja $Z(G, [w_1, w_2])$.
To se po prejšnji lemi zgodi natanko tedaj, ko besedi $w_1$ in $w_2$ generirata prosto grupo ranga 1, torej sta periodični z isto osnovo. 
Komutatorska lema nam podaja konstrukcijo, ki preprečuje pojav takšnih zapletov. V sledeči obliki se pojavi v magistrskem delu \cite{Schneider_2016} in članku \cite{Kozma_Thom_2016}.

% TODO malo drugače ubesedi dokaz

\begin{lema}
    \label{lem_komutatorska_lema}
     Naj bo $k \ge 2$, $e \in  \mathbb{N}$ in naj bodo besede $w_1, \ldots, w_m \in F_k$ netrivialne, pri čemer je $m = 2^{e}$. Potem obstaja beseda $w \in F_k$
     dolžine \begin{equation*}
     l(w) \le 2m \left(m + \sum_{i=1}^{m} l(w_{i}) \right),
     \end{equation*}  
    ki ni netrivialna potenca, da za vsako grupo $G$ velja \begin{equation*}
    Z(G, w) \supseteq \operatorname{Z}(G, w_1) \cup \ldots \cup \operatorname{Z}(G, w_m).
    \end{equation*}       
\end{lema}
    \begin{dokaz}
        Dokaz poteka z indukcijo po $e \in  \mathbb{N}$. Naj bo $F_k = \langle a_1, \ldots , a_{k} \rangle = \langle S \rangle$.  Za $e = 0$ (oziroma $m = 1$) vzamemo $w = [s, w_1]$, kjer je $s \in S$ takšna črka, da beseda $w_1$ ni potenca z osnovo $s$. 
        To lahko zaradi pogoja $k \ge 2$ vedno storimo . Zaradi ustrezne izbire je komutator $[s, w_1]$ aperiodičen z dolžino največ $2(l(w_1)  + 1)$.
        Kot smo videli v predhodnem razmisleku, za poljubno grupo $G$ velja $\operatorname{Z}(G, w) \supseteq \operatorname{Z}(G, s) \cup \operatorname{Z}(G, w_1)$.
        
        Zdaj se lotimo indukcijskega koraka v primeru $e \ge 1$ oziroma $m \ge 2$. Naj bodo podane besede $w_1, \ldots, w_{m / 2}, w_{m / 2 + 1}, \ldots, w_{2m}$. Po indukcijski predpostavki obstajata aperiodični besedi $v_1, v_2 \in  F_{k}$,  da velja
        \begin{equation*}
        l(v_1) \le m \left(\frac{m}{2} + \sum_{i=1}^{m / 2} l(w_{i}) \right), \,\,\, l(v_2) \le m \left(\frac{m}{2} + \sum_{i= m / 2 + 1}^{m} l(w_{i}) \right)
        \end{equation*}  
        in \begin{equation*}
        \operatorname{Z}(G, v_1) \supseteq \operatorname{Z}(G, w_1) \cup \ldots \cup \operatorname{Z}(G, w_{m / 2}),
        \end{equation*}  
        \begin{equation*}
            \operatorname{Z}(G, v_2) \supseteq \operatorname{Z}(G, w_{m / 2 + 1}) \cup \ldots \cup \operatorname{Z}(G, w_{m})
        \end{equation*}  
        za vsako grupo $G$.

         % TODO Schutzenbergova lema ...

        Zdaj moramo le še ugotoviti, kako lahko besedi $v_1$ ter $v_2$ ustrezno združimo. Po lemi \ref{lem_posledica_nielsen_schreier} vemo, da bo komutator $[v_1, v_2]$ trivialen natanko v primeru $v_1 = v_2^{\pm 1}$, ker sta $v_1$ in $v_2$ po predpostavki aperiodični. V primeru, da sta periodični, imamo \begin{equation*}
        \operatorname{Z}(G, w_1) = \operatorname{Z}(G, w_2)
        \end{equation*}  
         in lahko nastavimo $w := v_1$ ali $w := v_2$, pri čemer je pogoj na dolžino besede $w$ očitno izpolnjen. Če imamo $v_1 \neq v_2^{\pm 1}$, nastavimo $w := [v_1, v_2]$. V tem primeru po Schutzenbergovi lemi (TODO) beseda $w$ aperiodična. Indukcijska predpostavka nam zagotavlja 
         \begin{equation*}
         l(w) \le 2m  \left(\frac{m}{2} + \sum_{i=1}^{m / 2} l(w_{i}) \right) + 2m \left(\frac{m}{2} + \sum_{i= m / 2 + 1}^{m} l(w_{i}) \right) = 2m \left( m + \sum_{i = 1}^{m} l(w_{i}) \right).
         \end{equation*}
    \end{dokaz}

Lemo brez težav posplošimo tudi na število besed, ki ni dvojiška potenca.
\begin{lema}
\label{lem_komutatorska_lema_splosna}
Naj bo $k \ge 2$ in naj bodo podane netrivialne besede $w_1, \ldots, w_{m} \in  F_m$. Potem obstaja aperiodična beseda $w \in F_k$ dolžine \begin{equation*}
l(w) \le 8m \left(m +  \sum_{i = 1}^{m} l(w_{i}) \right),
\end{equation*}  
da za vsako grupo $G$ velja \begin{equation*}
\operatorname{Z}(G, w) \supseteq \operatorname{Z}(G, w_1) \cup \ldots \cup  \operatorname{Z}(G, w_{m}).
\end{equation*}    
\end{lema}

% To lemo sem v praktično enaki obliki uspel dokazati na bolj elementaten način v obliki leme \ref{lem_komutatorska_lema_splosna}, brez uporabe Nielsenovega izreka.
% Ideja za dokaz leme \ref{lem_komutaroska_lema_nova} izvira iz dokaza leme 2 v \cite[str.~7--8]{Schneider_2016}.  

% \begin{lema}
% \label{lem_komutaroska_lema_nova}
% Naj bo $G$ grupa, $H_i \subseteq  G$ njene simetrične podmnožice ($H_i = H_i ^{-1}$), in naj bo za vsak $i  \in \{1, \ldots, 2^{e}\}$ beseda $w_{i} \in F_k$ zakon v podmnožici $H_i$. Potem obstaja beseda $w \in F_k$ dolžine \begin{equation*}
% l(w) \le  m \sum_{i = 1}^{n} l(w_i),
% \end{equation*}  
% za katero velja $H_1 \cup  \ldots \cup H_m \subseteq  Z(G, w)$, torej je zakon v vseh podgrupah $H_i$. 
% \end{lema}
% \begin{dokaz}
% Dokaz poteka z indukcijo po $e \in  \mathbb{N}$ za primer $k = 2$, za $k \ge 3$ je dokaz praktično enak oziroma lažji. Naj bo $F_2 = \langle a , b \rangle$. Za $e = 0$ (oziroma $m = 1$) vzamemo zadostuje $w$, njena dolžina je $l(w_1)$, za poljubno grupo $G$ velja $\operatorname{Z}(G, w) \supseteq \operatorname{Z}(G, w_1)$. 
%         Zdaj se lotimo indukcijskega koraka v primeru $e \ge 1$ oziroma $m \ge 2$. Naj bodo podane besede $w_1, \ldots, w_{m / 2}, w_{m / 2 + 1}, \ldots, w_{m}$. Po indukcijski predpostavki obstajata besedi $v_1, v_2 \in  F_{2}$, da velja
%         \begin{equation*}
%         l(v_1) \le \frac{m}{2} \sum_{i=1}^{m / 2} l(w_{i}) , \,\,\, l(v_2) \le \frac{m}{2} \sum_{i= m / 2 + 1}^{m} l(w_{i})
%         \end{equation*}
%         in \begin{equation*}
%         \operatorname{Z}(G, v_1) \supseteq \bigcup_{i = 1}^{m / 2} H_i^2, \,\,\,  \operatorname{Z}(G, v_2) \supseteq \bigcup_{i = m / 2 + 1}^{m} H_i^2,
%         \end{equation*}  
%         za vsako grupo $G$. 
%         Zdaj moramo le še utemeljiti, da lahko besedi $v_1$ ter $v_2$ ustrezno združimo. Če bi takoj definirali $w = [v_1, v_2]$, bi v primeru $v_1 = v_2^{\pm 1}$ namreč dobili trivialno besedo, česar si ne želimo.
%         Zato obravnavajmo besedi $v_1$ in $v_2$ glede na to, s katero črko se začneta oziroma končata. Za lažjo notacijo bomo pisali, da je beseda element $V_{s_1 s_2}$, če se začne s črko $s_1 \in \left\{ a^{\pm 1}, b^{\pm 1} \right\}$ in konča s črko $s_2 \in \left\{ a^{\pm 1}, b^{\pm 1}\right\}$.   
%         Uvedimo še bijekciji $\tau, \kappa: F_2 \to F_2$, porojeni s predpisi $\tau: a \mapsto a^{-1} , b \mapsto b$ in $\kappa: a \mapsto b, b \mapsto a$. Ni težko preveriti, da z njima lahko vse besede prevedemo na eno izmed oblik $V_{ab}$, $V_{aa}$ ali $V_{aa^{-1}}$. Na primer za besedo $ba^{-1} \in V_{ba^{-1}}$ velja $\tau(\kappa(ba^{-1})) = ab \in V_{ab}$ in podobno za ostale.
%         Najprej besedi $v_1$ in $v_2$ prevedemo na besedi $v_1'$ in $v_2'$, ki sta ene izmed treh oblik. Nato ju z ustrezno uporabo preslikav $\tau$ in $\kappa$ pretvorimo v besedi $v_1''$ in $v_2''$ tako, da komutator $w = [v_1'', v_2'']$ gotovo ne bo trivialen.
%         \begin{enumerate}
%             \item V primeru $v_1', v_2' \in V_{aa } \cup V_{a a^{-1}}$ nastavimo $v_{2}'' = \kappa(v_2)$.
%             \item V primeru $v_1' \in V_{aa}, v_2' \in V_{ab}$ ne pride do krajšanja, imamo namreč \begin{equation*}
%                 [v_1', v_2'] =  [a \ldots a, a \ldots b] = a \ldots a a \ldots b a^{-1} \ldots a^{-1} b ^{-1} \ldots a.
%             \end{equation*}  
%             \item V primeru $v_1' \in V_{ab}, v_2' \in V_{aa}$ ne pride do krajšanja, imamo namreč  \begin{equation*}
%                 [v_1', v_2'] = [a \ldots b, a \ldots a] = a \ldots b a \ldots a b^{-1} \ldots a^{-1} a ^{-1} \ldots a^{-1}.
%             \end{equation*}
%             \item V primeru $v_1' \in V_{aa^{-1}}, v_2' \in V_{ab}$ nastavimo $v_2'' = \kappa(v_2)$, imamo namreč \begin{equation*}
%                 [v_1', v_2''] =  [a \ldots a^{-1}, b \ldots a] = a \ldots a^{-1} b \ldots a a \ldots a^{-1} a ^{-1} \ldots b^{-1}.
%             \end{equation*}
%             \item V primeru $v_1' \in V_{ab}, v_2'' \in V_{aa^{-1}}$ nastavimo $v_2'' = \tau(v_2)$, imamo namreč \begin{equation*}
%                 [v_1', v_2''] =  [a \ldots b, a^{-1} \ldots a] = a \ldots b a^{-1} \ldots a b^{-1} \ldots a^{-1} a ^{-1} \ldots a.
%             \end{equation*}
%             \item V primeru $v_1', v_2'' \in V_{ab}$ nastavimo $v_2'' = \kappa(v_2)$, imamo namreč \begin{equation*}
%                 [v_1', v_2''] =  [a \ldots b, b \ldots a] = a \ldots b b \ldots a b^{-1} \ldots a^{-1} a ^{-1} \ldots b^{-1}.
%             \end{equation*}  
%         \end{enumerate}
%         Po zgornjem razmisleku dobimo besedo oblike $w = [v_1'', v_2'']$, treba je le še razmisliti, da velja $Z(G, w) \supseteq \bigcup_{i = 1}^{m} H_i^2$. Po indukcijski prepodstavki vemo, da je \begin{equation*}
%         Z(G, v_1) \supseteq \bigcup_{i = 1}^{m / 2} H_i^2, \,\,\,  Z(G, v_2) \supseteq \bigcup_{i = m / 2 + 1}^{m} H_i^2,
%         \end{equation*}  
%         velja pa tudi \begin{equation*}
%             Z(G, v_1'') \supseteq \bigcup_{i = 1}^{m / 2} H_i^2, \,\,\,  Z(G, v_2'') \supseteq \bigcup_{i = m / 2 + 1}^{m} H_i^2,
%         \end{equation*}  
%           saj za poljubno besedo $u \in F_2$, ki je zakon v simetrični podmnožici $H_u \subseteq G$, velja (zaradi simetričnosti $H_u$) $Z(G, u) \cap  Z(G, \tau(u)) \supseteq H_u ^2$, hkrati pa vedno velja $Z(G, u) = Z(G, \kappa(u))$.
%         Z besedami, preslikavi $\tau$ in $\kappa$ ohranjata lastnost, da je beseda zakon v simetrični podmnožici grupe. 
%           Indukcijska predpostavka nam zagotavlja 
%          \begin{equation*}
%          l(w) \le m  \sum_{i=1}^{m / 2} l(w_{i}) + m \sum_{i= m / 2 + 1}^{m} l(w_{i})  = m \sum_{i = 1}^{m} l(w_{i}) .
%          \end{equation*}
% \end{dokaz}

% \begin{lema}
% \label{lem_komutatorska_lema_splosna}
% Naj bo $G$ grupa in naj bodo $w_{i} \in F_k$ netrivialne besede za $i \in \left\{ 1, \ldots, m \right\}$. Potem obstaja aperiodična beseda $w \in F_k$ dolžine \begin{equation*}
%     l(w) \le  8 m (m + \sum_{i = 1}^{m} l(w_i)),
%     \end{equation*}  
%     za katero velja $Z(G, w_1) \cup  \ldots \cup Z(G, w_m) \subseteq  Z(G, w)$.
% \end{lema}


\begin{dokaz}
    Naj bo $e$ takšno naravno število, da velja $m \le 2^{e} < 2m$. Nastavimo \begin{equation*}
    w_1' := w_1, \ldots, w_{m}' := w_{m}, w_{m+1}' := w_1, \ldots, w_{2^{e}}' := w_{2^{e} - m}. 
    \end{equation*}
    Ker velja $m < 2m$ in $\sum_{i = 1}^{2^{e}} w_i' \le 2 \sum_{i = 1}^{m} l(w_i)$, želena ocena sledi z uporabo leme \ref{lem_komutatorska_lema}.  
\end{dokaz}


Ta rezultat lahko nekoliko omilimo, da dobimo bolj praktično oceno.
\begin{posledica}
\label{psl_komutatorska_lema_prakticna}
Naj bo $k \ge 2$ in naj bodo podane netrivialne besede $w_1, \ldots, w_m \in  F_k$. Potem obstaja aperiodična beseda $w \in  F_k$ dolžine \begin{equation*}
l(w) \le 8m^2 (1 +  \max_{i = 1, \ldots, m} l(w_i) )
\end{equation*}      
\end{posledica}
\begin{dokaz}
    To je direktna posledica leme \ref{lem_komutatorska_lema_splosna} skupaj z dejstvom, da je \begin{equation*}
    \sum_{i = 1}^{m} l(w_{i}) \le m \max_{i = 1, \ldots, m} l(w_i).
    \end{equation*}  
\end{dokaz}

% TODO tukaj lahko komentiraš nekaj v zvezi z netrivialno potenco

\begin{primer}
Najelegantnejša uporaba komutatorske leme se pojavi pri obravnavi grupe kot direktni produkt svojih podgrup. Naj bo recimo $G = C_5 \times D_{10}$. 
V podgrupi $C_5$ imamo zakon $x^{5} \in F_2$ dolžine $5$, razširitvena lema \ref{lem_razsiritvena_lema} pa nam bo povedala, da obstaja zakon dolžine $8$ v $D_{10}$. Po prejšnji lemi torej obstaja
netrivialna beseda $w \in F_2$ dolžine $2 \cdot (5 + 8) = 26$.
\end{primer}

Čeprav je ta primer enostaven, je ključ do praktično vseh konstrukcij zakonov za družine, kar bomo videli recimo na koncu razdelka \ref{sec_grupe_psl2q} pri obravnavi družine grup $\operatorname{PSL}_2(q)$. 

% Definirajmo namreč grupo \begin{equation*}
% \Gamma_n = \prod_{G \text{ grupa},\, \lvert G \rvert \le n }  G 
% \end{equation*}  
  


\subsection{Razširitvena lema}

Nekoliko bolj povezana s strukturo grup je razširitvena lema. Za njeno formulacijo najprej definirajmo kratka eksaktna zaporedja.
\begin{definicija}
\label{def_kratko_eksaktno_zaporedje}
Naj bodo $A, B, C$ grupe in naj $\mathbf{1}$ označuje trivialno grupo. Kratko eksaktno zaporedje je zaporedje homomorfizmov
\begin{equation*}
\mathbf{1} \to A \xrightarrow{\varphi} B \xrightarrow{\psi} C \to \mathbf{1},
\end{equation*}  
kjer je $\ker \psi = \operatorname{im} \varphi$, $\varphi$ je injektivni in $\psi$ surjektivni homomorfizem.
\end{definicija}

\begin{lema}\label{lem_razsiritvena_lema}
    Naj bo \begin{equation*}
        \mathbf{1} \to N \to G \to G / N \to  \mathbf{1}  
    \end{equation*}  
    kratko eksaktno zaporedje grup. Naj bo $F_k = \langle a_1, \ldots, a_k \rangle  = \langle S \rangle$. Naj bo $w_N \in F_k$ netrivialni zakon za $N$ in $w_{ G / N} \in  F_k$ netrivialni zakon za $G / N$. 
    Potem obstaja netrivialni $k$-črkovni zakon za grupo $G$ dolžine kvečjem $l(w_N) l( w_{ G / N } )$. Od tod sledi \begin{equation*}
    \operatorname{girth}_{k}(G) \le \operatorname{girth}_{k}(N) \operatorname{girth}_{k}( G / N ). 
    \end{equation*}  
\end{lema}
\begin{dokaz}
    Dokaz obravnava dva možna primera oblike zakona za $w_{ G / N }$. 
    \begin{enumerate}
        \item Če je $w_{ G / N } = s^{n}$ za neki $s \in S$, $n \in  \mathbb{Z} \setminus \left\{ 0\right\} $, so vse besede oblike $t^{n}$, kjer je $t \in S$, zakoni za $G / N$. Zato lahko vzamemo besedo \begin{equation*}
        w = w_N(a_1^{n}, \ldots, a_{k}^{n}),
        \end{equation*}  
        ki je netrivialni zakon za $G$. To sledi iz dejstev, da preslikava $g \mapsto g^{n}$ slika $G$ v $N$ (ker je $s^{n}$ zakon za $G / N$), $w_N$ pa je netrivialni zakon za $N$.    
    \item Sicer lahko brez škode za splošnost predpostavimo, da je zakon oblike $w_{ G / N} = a_1 w_{ G / N}' a_2$, kjer se $w_{ G / N}'$ niti ne začne z $a_1^{-1}$ niti konča z $a_2^{-1}$. To smemo storiti, ker lahko na besedi uporabimo ciklino rotacijo in vzamemo 
    $w_{ G / N}'' = s w' t$, pri čemer $s, t \in  S \cup  S^{-1}$ in $st \neq 1$. Če je potrebno, lahko na tej besedi uporabimo avtomorfizem grupe $F_k$, ki ga inducirajo $s \mapsto a_1$, $a_1 \to s$, $a_2 \mapsto t$, $t \mapsto a_2$ in $r \mapsto r$ za vse ostale $r \in S$.  
   Nato definiramo besede \begin{equation*}
   w_i := w_{ G / N}(a_{i}, \ldots, a_{k}, a_1, \ldots, a_{i - 1}). 
   \end{equation*}  
   Ni težko preveriti, da so netrivialne kombinacije besed $w_i$, torej $w_{i} w_{j}$, $w_{i}^{-1} w_{j}$, $w_{i}^{-1} w_{j}^{-1}$, $w_{i} w_{j}^{-1}$, $w_{i} w_{i}$, $w_{i}^{-1} w_{i}^{-1}$ za vse $i,j \in \left\{ 1, \ldots, k \right\}, \, i \neq j$, v okrajšani obliki.
   Zato je \begin{equation*}
   w := w_N (w_1, \ldots, w_{k})
   \end{equation*}  
    netrivialni zakon za grupo $G$. Vse besede $w_{i}$ namreč inducirajo preslikave, ki $G$ slikajo v $N$, $w_N$ pa je netrivialni zakon za $N$.       
    \end{enumerate}
\end{dokaz}

\begin{primer}
    S pomočjo te leme lakho dokažemo, da za vsak $n \in \mathbb{N}$ obstaja zakon $w \in F_2$ v diedrski grupi $D_{2n}$ dolžine $8$. Ker ima podgrupa $\langle r \rangle \subseteq D_{2n}$ indeks 2, je edinka,
    zato lahko tvorimo kratko eksaktno zaporedje \begin{equation*}
        \mathbf{1} \to \langle r \rangle  \xrightarrow{\varphi} D_{2n} \xrightarrow{\psi} D_{2n} / \langle r \rangle  \to \mathbf{1}.
        \end{equation*}
        Ker je podgrupa $\langle r \rangle$ Abelova, je v njej zakon beseda $[a, b] = aba^{-1} b^{-1}$, grupa $D_{2n} / \langle r \rangle$ pa je moči $2$ in je zato v njej zakon beseda $a^2$. Po lemi \ref{lem_razsiritvena_lema} torej obstaja beseda $w \in F_2$ dolžine $l(w) \le  8$, ki je zakov v $D_{2n}$.
        Če sledimo konstrukciji izreka vidimo, da je to natanko beseda $w = [a^2, b^2] = a^2 b^2 a^{-2} b^{-2}$.   
\end{primer}

Iz primera je razvidno, da je moč razširitvene leme je še posebej izrazita, kadar ima grupa kakšno edinko z lepimi lastnosti, kot so na primer rešljivost, nilpotentnost ali celo Abelovost. Od tod sledi tudi, da so s stališča obravnave virtualno nilpotentne oziroma rešljive grupe -- torej grupe, ki imajo nilpotentno oziroma rešljivo edinko končnega indeksa --
praktično enake nilpotentnim oziroma rešljivim. Naravna posledica razširitvene leme je tudi dejstvo, da bistveno vlogo pri iskanju kratkih zakonov igrajo enostavne grupe, saj lahko problem iskanja zakonov v neenostavnih grupah vedno prevedemo na dva manjša; na problema edinke in njenega kvocienta.  
