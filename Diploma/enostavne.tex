\section{Enostavne, polenostavne in simetrične grupe}

Na prvi pogled se zdi nenavadno obravnavati enostavne in simetrične grupe v istem poglavju. Po strukturi se namreč močno razlikuejo; simetrične grupe imajo bogato strukturo edink (kar nam kažejo recimo izreki Sylowa ? TODO ali res), po drugi strani pa enostavne
nimajo nobenih pravih netrivialnih. Razlog za takšno obravnavo se skriva v postopku za iskanje kratkih zakonov, ki poteka z uporabo naključnih sprehodov. Ta postopek ni konstruktiven, zgolj pokaže nam obstoj nekega kratkega zakona v grupi, čeprav njegove konkretne oblike ne poznamo.
Naključni sprehodi so se izkazali za ključno orodje pri obravnavi družine enostavnih grup $\operatorname{PSL}_2(q)$, ki bo glavna tema poglavja.

Začnimo z razmislekom o pomembnosti enostavnih grup pri iskanju kratkih zakonov v splošnih grupah. Glavno idejo smo pravzaprav že videli v opombi pod razširitveno lemo \ref{lem_razsiritvena_lema}, kjer smo ugotovili, da
lahko problem iskanja kratkih zakonov v neki konkretni grupi prevedemo na problem o njeni edinki in kvocientu po tej edinki. To idejo bomo povezali z našim znanjem o rešljivih grupah z uvedbo rešljivega radikala.

\begin{definicija}
\label{def_resljiv_radikal}
Naj bo $G$ končna grupa. Največjo rešljivo edinko $G$ imenujemo rešljivi radikal grupe $G$ in ga označimo z $S(G)$. Če je $S(G) = \mathbf{1}$, rečemo, da je $G$ polenostavna grupa.
\end{definicija}
\begin{lema}
\label{lem_dobra_definiranost_resljivega_radikala}
Rešljivi radikal je dobro definiran za končne grupe.
\end{lema}
\begin{dokaz}
    Naj bosta $M$ in $N$ rešljivi edinki končne grupe $G$. Po četrti točke trditve \ref{trd_lastnosti_resljivih_grup} je tudi $MN$ rešljiva edinka (produkt edink je vedno edinka, manj očitna je rešljivost). Ker je grupa $G$ končna, ima kočno mnogo edink,
    s primerjanjem vseh parov v končnem številu korakov najdemo največjo. 
\end{dokaz}

\begin{lema}
\label{lem_resljiv_radikal_je_polenostaven}
Naj bo $G$ končna grupa. Potem je kvocient $G / S(G)$ polenostavna grupa. 
\end{lema}
\begin{dokaz}
    Dokaz poteka s protislovjem. Recimo, da $G / S(G)$ ni polenostavna grupa in ima netrivialno rešljivo edinko $N$. Po korespondenčnem izreku je $N = N' / S(G)$ za neko edinko $N' \triangleleft G$.
    Po tretji točki trditve \ref{trd_lastnosti_resljivih_grup} sledi, da je $N'$ rešljiva in hkrati strogo večja od $S(G)$, kar je protislovno z definicijo rešljivega radikala.
\end{dokaz}

Naj bo $G$ poljubna končna grupa. S tvorjenjem kratkega eksaktnega zaporedja \begin{equation*}
\mathbf{1} \to S(G) \to G \to  G / S(G) \to  \mathbf{1}
\end{equation*}  
in uporabo razširitvene leme \ref{lem_razsiritvena_lema} vidimo, da za netrivialna zakona $w_{S(G)}$ in $w_{G / S(G)}$ v grupah $S(G)$ oziroma $G / S(G)$ obstaja netrivialni zakon $w_G$ v grupi $G$, dolžine \begin{equation*}
l(w_G) \le  l(w_{S(G)}) l (w_{G / S(G)}).
\end{equation*}  

Na straneh 28--31 vira \cite{Schneider_2016} je podan razmislek, kako problem v polenostavnih grupah prevedemo na problem o simetričnih grupah in grupah avtomorfizmov enostavnih grupa. Slednjih se lahko presenetljivo elegantno lotimo s pomočjo Schreierjeve domneve, ki jo bomo formulirali.

\begin{definicija}
\label{def_grupa_zunanjih_avtomorfizmov}
Naj bo $G$ grupa in $\operatorname{Aut}(G)$ njena grupa avtomorfizmov. Znano dejstvo je (TODO enovrstični račun), da je grupa notranjih avtomorfizmov $\operatorname{Inn}(G) = \left\{ x \mapsto g x g^{-1}  \middle|\,  g \in G  \right\}$ njena edinka.
Kvocientu $\operatorname{Out}(G) :=  \operatorname{Aut}(G)  /  \operatorname{Inn}(G)$ rečemo grupa zunanjih avtomorfizmov grupe $G$. 
\end{definicija}

\begin{izrek}
\label{izr_Schreierjeva_domneva}
 Naj bo $G$ končna enostavna grupa, ki ni Abelova. Potem je grupa $\operatorname{Out}(G)$ rešljiva razreda največ $3$.
\end{izrek}

To domnevo so potrdili z uporabo klasifikacije končnih enostavnih grup. Vprašanje, ali obstaja bolj elementaren dokaz, je še vedno odprto. % TODO kje je vir za to  
Glede na to, da se iskanje zakonov v enostavnih grupah močno naslanja na to klasifikacijo, bomo domnevo brez zadržkov uporabili. Naj bo $H$ poljubna enostavna grupa. 
S tvorjenjem kratkega eksaktnega zaporedja \begin{equation*}
    \mathbf{1} \to \operatorname{Inn}(H)  \to \operatorname{Aut}(H)  \to  \operatorname{Out}(H)  \to  \mathbf{1}
\end{equation*}  
in uporabo razširitvene leme \ref{lem_razsiritvena_lema} vidimo, da za netrivialna zakona $w_{\operatorname{Inn}(H) }$ in $w_{\operatorname{Out}(H) }$ v grupah $\operatorname{Inn}(H) $ oziroma $\operatorname{Out}(H) $ obstaja netrivialni zakon $w_{\operatorname{Aut}(H) }$ v grupi $\operatorname{Aut}(H) $, dolžine \begin{equation*}
    l(w_{\operatorname{Aut}(H) }) \le  l(w_{\operatorname{Inn}(H) }) l (w_{ \operatorname{Out}(H) }).
    \end{equation*}    
Ker je $H$ enostavna, velja $H \cong \operatorname{Inn}(H)$. Za splošno grupo $G$ namreč velja $ G / Z(G) \cong \operatorname{Inn}(G)$, v primeru enostavnosti (nekomutativne) grupe pa je center seveda trivialen. Dalje, po Schreierjevi domnevi \ref{izr_Schreierjeva_domneva} in lemi \ref{lem_vrednost_cn} obstaja zakon dolžine $c_3 = 50$, ki je zakon za vse rešljive grupe razreda $3$ ali manj.
Tako zgornjo enačbo prevedemo na \begin{equation*}
    l(w_{\operatorname{Aut}(H) }) \le  50 l(w_H).
\end{equation*}  

% TODO v resnici je bolj smiselno, če ta sklep spada na konec poglavja, saj skupaj poveže vse posamezne ugotovitve + pojmi niso tako neznani

Ko vse to združimo, dobimo (TODO napiši po komponentah kaj dobiš). Zdaj se lotimo posameznih delov te enačbe. Ker je podrobna obravnava spošnih enostavnih grup in simetričnih grup preobsežna za okvir te diplomske naloge, bomo zgolj navedli glavne rezultate in povzeli njihove dokaze.

\subsection{Simetrične grupe}

% opis članka o Thom - Kozma, napišeš v čem se razlikuje postopek z naključnimi sprehodi
% opišeš in kometiraš Liebeckov izrek
% če boš napisal konkretno Liebeckov izrek, je treba definirati venčni produkt 
% (morda malo zoprno, ampak ne bi smelo biti prehudo) 

Obravnava simetričnih grup je najnatančneje opisana v članku \cite{Kozma_Thom_2016}, kjer avtorja dokažeta obstoj kratkih zakonov v simetričnih grupah s pomočjo naključnih sprehodov.
Ker je celoten dokaz glavnega rezultata preveč specifičen za okvir te diplomske naloge, bom predstavil le del, ki je bralcu te naloge vsebinsko nov, tematsko drugačen od dosedanjih konstrukcij s komutatorsko in razširitveno lemo.

Osnovna ocena dolžin kratkih zakonov v simetričnih grupah izhaja iz zgornje meje maksimalnega reda elementov v simetrični grupi,
ki jo je dokazal Edmund Landau leta 1903 v knjigi \cite{Landau_1903}: \begin{equation*}
\max_{\sigma \in S_n} \text{ord}(\sigma) \le \exp((1 + o(1)) (n \log n)^{1 / 2}).
\end{equation*}  
Od tod po enakem postopku kot na koncu razdelka \ref{sec_grupe_psl2q} z uporabo komutatorske leme $a, a^2, \ldots, a^{\max_{\sigma \in S_n} \text{ord}(\sigma)}$ na elementih proste grupe $F_2 = \langle a, b \rangle$ dobimo asimptotsko gledano enako oceno \begin{equation*}
    \alpha(n)  \le \exp((1 + o(1)) (n \log n)^{1 / 2}),
\end{equation*}  
kjer smo z $\alpha(n)$ označili dolžino najkrajšega netrivialnega zakona v grupi $S_n$.

Avtorja članka \cite{Kozma_Thom_2016} sta rezultat močno izboljšala in sicer na obliko \begin{equation}\label{eq_kozma_thom}
    \alpha(n)  \le \exp((1 + o(1)) \log(n)^4 \log (\log n))
\end{equation}
z uporabo: 
\begin{itemize}
    \item Liebeckovega izreka (\cite{Liebeck_1984}) o strukturi podgrup grupe $S_n$, ki opredeli vrste podrgup v odvisnosti od načina delovanja na $S_n$. Najpomembnejši rezultat izreka je ugotovitev, da je vsaka podgrupa $\Gamma \subseteq  S_n$, ki ne sodi med prve štiri vrste, omejena z $\lvert \Gamma \rvert \le \exp((1 + o(1)) \log(n)^2)$. 
    \item Helfgott--Seressov izrek (\cite{Helfgott_Seress_2013}), ki poda asimptotsko oceno za diametre Cayleyjevih grafov grupe $S_n$.
\end{itemize} 
Oba rezultata sta zahtevna in temeljita na uporabi klasifikacije končnih enostavnih grup. Dokaz ocene \ref{eq_kozma_thom} v grobem poteka v dveh delih, in sicer za vsak $k \le n$ razdeli pare $(\sigma, \tau) \in S_k^2$ na tiste,
ki generirajo grupo $S_k$ ali $A_k$ (to je prva vrsta podgrup po Liebeckovem izreku) in na pare, ki generirajo preostale vrste podgrup.
\begin{enumerate}
    \item Najprej za vsako naravno število $k \le n$ s $P(k)$ označimo množico $k$-ciklov grupe $S_k$. Helfgott-Seressov izrek nam zagotovi obstoj množice $W \subseteq F_2$, velikosti $\lvert W \rvert \le 8n^2 \log n$, da za vsak $w \in W$ velja \begin{equation}\label{eq_helfgot_ocena}
        l(w) \le \exp((1 + o(1)) \log(n)^{4} \log(\log(n))).
    \end{equation}  
    Še več, za vse $k \le n$ in vse pare $(\sigma, \tau) \in S_k^2$, ki generirajo $S_k$,
    obstaja beseda $w \in W$, tako da je $w(\sigma, \tau) \in P(k)$. Ker beseda $1_{F_2}$ ni $k$-cikel (za $k \ge 2$, primer $k = 1$ pripada trivialni podgrupi in nas ne zanima), je beseda $w$ netrivialna. Nato definiramo množico \begin{equation*}
    W' = \left\{ w^{k}  \middle|\,  w \in W , \, 1 \le  k \le  n \right\}, 
    \end{equation*}  
    ki ne vsebuje enote $1_{F_2}$, ker je grupa $F_2$ torzijsko prosta (TODO skliči se na dokaz te trditve v uvodu, velikokrat se ga posredno uporablja). S pomočjo ocene moči $W$ sklepamo $\lvert W' \rvert \le 8 n^3 \log n$.
    Ker za vsak $k \le n$ in za vsak $(\sigma, \tau) \in S_k^2$ obstaja beseda $w \in W'$, da je $w(\sigma, \tau) = 1_{F_2}$, po komutatorski lemi \ref{psl_komutatorska_lema_prakticna} in oceni \ref{eq_helfgot_ocena} obstaja netrivialna beseda $v \in F_2$, dolžine
    \begin{equation*}
    l(v) \le \exp((1 + o(1)) \log(n)^{4} \log(\log(n))).
    \end{equation*}  
    \item V drugem primeru z obravnavanjem podgrup po Liebeckovem izreku konstruiramo netrvialno besedo $\tilde{v} \in F_2$, ki trivializira vse pare $(\sigma, \tau) \in S_k^2$, ki ne generirajo
    grupe $S_k$ (veljati mora $\tilde{v}(\sigma, \tau) = 1_{F_2}$ za vse pare s to lastnostjo). Na primer, v prvo vrsto spadajo podgrupe oblike $S_k$ ali $A_k$, ki spadajo pod prejšnjo točko dokaza, mejo za meto vrsto pa nam direktno podaja Liebeckov izrek. Vrste dva do štiri je treba obravnavati
    vsako posebej. Na koncu zakone za posamezne vrste grup povežemo s komutatorsko lemo.    
\end{enumerate}

Avtorja sta razdelek končala z mislijo (\cite[str.~82]{Kozma_Gady_2016}), da po Babaijevi domnevi sledi po praktično enakem dokazu ocena \begin{equation*}
\alpha(n) \le \exp((1 + o(1)) \log{n} \log(\log (n))) = n^{(1 + o(1)) \log(\log(n))}.
\end{equation*}  

\begin{trditev}
\label{trd_babaijeva_domneva}
 TODO, vprašaj Jezernika za vir ? Citiraj predstavitev ?? 
\end{trditev}

\subsection{Enostavne grupe}

% kratek opis članka https://arxiv.org/abs/1811.05401v2
% kako ta članek vpliva na grupe PSL_2(q)

TODO dodaj razmislek o sporadičnih in enostavnih grupah in napiši nekaj o klasifikaciji končnih enostavnih grup

\subsection{Grupe $PSL_2(q)$}\label{sec_grupe_psl2q}

Tekom tega poglavja bo $p$ vedno označevalo praštevilo, $q$ pa praštevilsko potenco oblike $q = p^{k}$ za neko naravno število $k \ge $1. Začnimo z definicijo družine grup $\operatorname{PSL}_n(q)$.

\begin{definicija}\label{def_pslnq_in_psl2q}
    Naj bo $n \in \mathbb{N}$ in $q \in \mathbb{N}$ praštevilska potenca, torej $q = p^{k}$. Potem definiramo \begin{equation*}
        \operatorname{PSL}_n(q) = {\operatorname{SL}_n(q)} / {Z(\operatorname{SL}_n(q))}.
     \end{equation*}   
    V primeru $n = 2$ dobimo so elementi podgrupe $Z(\operatorname{SL}_n(q))$ skalarne $2 \times 2$ matrike oblike $\lambda I$ z lastnostjo $\det \lambda I = 1$. To enačbo prevedemo na enačbo oblike $(\lambda - 1)(\lambda + 1) = 0$.
    Če ima polje $\mathbb{F}_q$ karakteristiko $2$ -- kar se zgodi natanko v primeru $q = 2^{k}$ -- sta $\lambda_{1,2} = \pm 1$ isti element, sicer pa dva različna. Tako dobimo
    \begin{equation*}
                \operatorname{PSL}_2(q) = \begin{cases}
                    \operatorname{SL}_2(q); & p = 2,  \\
                    {\operatorname{SL}_2(q)} / {\left\{ I, -I \right\} }; & p \neq 2.
                \end{cases}
             \end{equation*}   
    \end{definicija}
    
    Družina $\operatorname{PSL}_2(q)$ ima -- poleg svoje problematičnosti pri iskanju kratkih zakonov -- zelo posebne lastnosti. Ena izmed glavnih je sledeča. 
    \begin{trditev}\label{trd_dolzina_zakonov_za_psl2p}
    Naj bo $p$ praštevilo. Potem ima vsak netrivialni zakon v grupi $\operatorname{PSL}_2(p)$ dolžino vsaj $p$.
    \end{trditev}
    \begin{dokaz}
        TODO, imaš v Schneiderju
    \end{dokaz}
    \noindent
    Direktna posledica te leme je recimo dejstvo, da grupa $\text{Sym}(\mathbb{N})$ nima netrivialnih zakonov, saj vsebuje vse $\operatorname{PSL}_2(p)$ kot podgrupe.
    Druga taka grupa je recimo $\operatorname{SL}_2(\mathbb{Z})$, saj vsebuje vse grupe $\operatorname{PSL}_2(q)$ kot kvociente (TODO pri Jezerniku je to za domačo nalogo). 
    Ker se zakoni prenašajo na kvociente, enako kot v prvem primeru sklepamo, da $\operatorname{SL}_2(\mathbb{Z})$ ne more imeti netrivialnih zakonov.   

    \subsubsection{Konstrukcija zakonov v grupah $PSL_2(q)$}

    Osnovna konstrukcija zakonov za grupe $\operatorname{PSL}_2(q)$ poteka prek obravnave redov elementov in uporabe komutatorske leme \ref{lem_komutatorska_lema_posplositev}.
    Dokaz je prirejen po \cite[str.~36--37]{Schneider_2016}. 
    \begin{lema}
    \label{lem_redi_elementov_v_psl2q}
    Red poljubnega element $A \in  \operatorname{PSL}_2(q)$ deli vsaj eno izmed števil $p$, $q-1$ ali $q + 1$. 
    \end{lema}
    \begin{dokaz}
    % (TODO popravi, da boš imel Jordanovo formo $J_A$ in ne matirke $A$)
    Naj bo matrika $A \in \operatorname{PSL}_2(q)$. Obravnavajmo primere glede na njeno Jordanovo formo. Naj bo $\chi_A(X) \in \mathbb{F}_q[X]$ karakteristični polinom matrike $A$. 
    \begin{enumerate}
        \item Če je $A$ diagonalizabilna, je oblike \begin{equation*}
        A \sim  \begin{bmatrix}
            \alpha & 0 \\
            0 & \beta \\
        \end{bmatrix},
        \end{equation*}  
          kjer sta $\alpha, \beta \in  \mathbb{F}_q^{*}$ (0 ne moreta biti, ker je matrika $A$ obrnljiva). Ker je $(\mathbb{F}_q^{*}, \cdot)$ grupa moči $q-1$, velja $\alpha^{q-1} = \beta^{q -1} = 1$ in od tod $A^{q-1} = I$.
          \item Če je $\chi_A(X)$ razcepen v $\mathbb{F}_q[X]$, vendar matrika $A$ ni diagonalizabilna, mora biti oblike \begin{equation*}
          A \sim  \begin{bmatrix}
            1 & \alpha\\
            0 & 1\\
          \end{bmatrix} = I + N.
          \end{equation*}  
        Diagonalna elementa morata namreč oba biti enaka $1$ po razmisleku v definiciji \ref{def_pslnq_in_psl2q}. Ker velja $A^{p} = (I + N)^{p} = I^{p} + N^{p} = I$, red matrike $A$ deli $p$.  
    \item Če $\chi_A(X)$ ni razcepen v $\mathbb{F}_q[X]$, je razpecen v  $\mathbb{F}_{q^2}[X] = \mathbb{F}_q[X] / (\chi_A(X))$. Naj bo $\alpha \in \mathbb{F}_{q^2}$ neka ničla $\chi_A(X)$. Pokazati moramo, da je potem tudi $\alpha^q$ njegova ničla.
    Naj bo $\chi_A(X) = X^2 + bX + c$ za neka $b ,c \in \mathbb{F}_q^{*}$. Potem iz enačbe $\alpha^2 + b \alpha + c  = 0$ sledi \begin{equation*}
    0 = (\alpha^{q} + b \alpha + c)^q =  \alpha^{2q} + b^{q} \alpha^{q} + c^{q} =_\text{točka 1} \alpha^{2q} + b \alpha^{q} + c.    
    \end{equation*}  
    Tako lahko matriko $A$ diagonaliziramo v kolobarju $M_2(\mathbb{F}_{q^2})$ \begin{equation*}
    A \sim  \begin{bmatrix}
        \alpha & 0\\
        0 & \alpha^{q}\\
    \end{bmatrix}.
    \end{equation*}  
    Ker velja $\det A = \alpha \alpha^{q} = 1$, red $A$ deli število $q-1$.
    \end{enumerate}   
\end{dokaz}
    Za konkretno grupo $G = \operatorname{PSL}_2(q)$ definirajmo podmnožice \begin{equation*}
        H_m := \left\{ A \in \operatorname{PSL}_2(q)  \middle|\,  A^{m} = I \right\}
    \end{equation*}  
       za števila $m \in \left\{ p, q-1 , q+1\right\}$.
    Po razmisleku iz prejšnje leme te podmnožice tvorijo pokritje $G$. Z uporabo komutatorske leme \ref{lem_komutaroska_lema_nova}, je zakon v grupi $G$ beseda oblike \begin{align*}
        w &= [[b a^{p} b^{-1}, a^{q-1}], a^{q + 1}]  \\
         &= b a^{p} b^{-1} a^{q-1} b a^{-p} b^{-1} \cancel{ a^{1 - q}} a^{q + 1} \cancel{a^{q -1}} b a^{p} b^{-1} a^{1 - q} b a^{-p} b^{-1} a^{- q - 1} \\ 
         &= b a^{p} b^{-1} a^{q-1} b a^{-p} b^{-1}  a^{q + 1}  b a^{p} b^{-1} a^{1 - q} b a^{-p} b^{-1} a^{- q - 1} 
    \end{align*}  
    dolžine \begin{equation*} 
    l(w) = 4(2 + p + q) \le 8(q + 1). 
    \end{equation*}
    Pred uporabo komutatorske leme moramo navesti še dva rezultata.
    \begin{lema}
    \label{lem_velikost_grupe_psl2q}
    \begin{equation*}
        \lvert \operatorname{PSL}_2(q) \rvert   = \begin{cases}
            (q^2 - 1) q; & p = 2,  \\
            \frac{1}{2} (q^2 - 1) q ; & p \neq 2.
        \end{cases}
     \end{equation*} 
    \end{lema}
    \begin{dokaz}
    Grupa $\operatorname{GL}_2(q)$ ima $(q^2  -1)(q^2 - q)$ elementov. Če hočemo, da je matrika $A \in M_2(\mathbb{F}_q)$ obrnljiva, imamo namreč za prvi stolpec $q^2 -1$ izbir, za drugega pa $q^2 - q$.
    Od tod sledi, da ima $\operatorname{SL}_2(q)$ $(q^2  -1)q$ elementov, saj je $\lvert \mathbb{F}_q^{*} \rvert = q-1$. V primeru $p \neq 2$, nam kvocient po centru odbije še polovico elementov.   
    \end{dokaz}
    
    \begin{lema}
    \label{lem_gostota_prastevil}
    Naj bo preslikava $\tau : \mathbb{R} \to \mathbb{N} \cup \left\{ 0\right\}$, ki prešteje število praštevilskih potenc, podana s predpisom \begin{equation*}
    \tau(x) = \sum_{p^{k} \le x, \, k \in \mathbb{N}} 1.
    \end{equation*}  
     Potem velja $\tau(x) = (1 + o(1)) \frac{n}{\log(n)}$.     
    \end{lema}
    Ta lema je ena izmed oblik osnovnega izreka o praštevilih. Leta 1851 je Čebišev dokazal (\cite[str.~4--5]{Granville_1993}), da limita $\frac{\tau(x)}{x / \log(x)}$ -- če le obstaja -- mora biti 1, kar bi potrdilo Gaussovo domnevo. Obstoja limite mu ni uspelo dokazati, je pa to uspelo Riemannu v svojem znamenitem članku \cite{Riemann_1859} leta 1859, v katerem je povezal porazdeltev praštevil s funkcijo zeta in formuliral Riemannovo hipotezo.
    Ker je dokaz netrivialen, ga bomo opustili, Riemann ga je v prej omenjenem članku dokazal z uporabo kompleksne analize. 
    Nekoliko več o tej lemi piše v članku \cite{Kozma_Thom_2016}.            
    \noindent
    Zdaj se lahko lotimo konstrukcije netrivialnega zakona za vse grupe oblike $\operatorname{PSL}_2(q)$, moči manjše ali enake številu $n \in \mathbb{N}$. Z uporabo lem \ref{lem_velikost_grupe_psl2q} in \ref{lem_gostota_prastevil}
    vemo, da moramo moramo konstruirati zakone za vse grupe $\operatorname{PSL}_2(q)$, za katere je $q \le \sqrt[3]{(1 + o(1)) 2n}$. Od tod dobimo besedo $w \in F_2$ dolžine \begin{equation*}
    l(w) \le 8 \left( \frac{3 \sqrt[3]{(1 + o(1)) 2n}}{\log((1 + o(1)) 2n))}  \right)^2 \cdot 8 \sqrt[3]{(1 + o(1)) 2n} \le 1152 (1 + o(1)) \frac{n}{\log(n)^2},
    \end{equation*}  
    ki je zakon za vse grupe $\operatorname{PSL}_2(q)$, moči $n$ ali manj. Ta rezultat ni najboljši in je predstavljal oviro, kot je bilo omenjeno v tretjem odstavku članka \cite[str.~6]{Bradford_Thom_2017}.
    Izognemo se ji lahko z uporabo naključnih sprehodov.

    \subsubsection{Iskanje zakonov v grupah $PSL_2(q)$ z naključnimi sprehodi}
    

\cite{Kozma_Thom_2016}