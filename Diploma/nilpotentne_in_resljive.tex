\section{Nilpotentne in rešljive grupe}

% nekaj o centralnih in spodnjih zaporedjih (ekvivalenco definicij lahko daš v appendix)
% vse p-grupe so nilpotentne (klasična ekvivalenca z nilpotentnimi p-grupami)
% dodaj primere nilptotentnih grup (predvsem tiste, ki nastopajo v računalniškem delu!)
% rešljiv radikal, razmislek o razdelitvi splošnega problema na rešljive in polenostavne grupe

% napiši nekaj malega o tem, kako dolžina zakonov v grobem lahko predstavlja kompleksnost strukture grupe. Ene izmed najosnovnejših - poleg Abelovih - so nilpotentne in rešljive,
% slednje igrajo tudi pomembno vlogoo pri reševanju splošnega problema

% \begin{definicija}
% \label{def_centralna_vrsta}
% Naj bo $G$ grupa in $H$ njena podgrupa. Centralna vrsta je zaporedje njenih podgrup $(H_k)_{k \ge 1}$, za katerega velja $H_1 = H$ in $H / Z(G / H_{i + 1})$ oziroma ekvivalentno $[H_i, G] \subseteq H_{i + 1}$ za vsak $i \ge 1$.
% Rečemo, da se centralna vrsta izteče z grupo $K$, če obstaja naravno število $n$, da velja $H_k = K$ za vsako naravno število $k \ge n$.
% \end{definicija}

\begin{definicija}
\label{def_iztek_zaporedja}
Naj bo $G$ grupa in $(H_k)_{k \ge 1}$ padajoče zaporedje njenih podgrup, torej $H_{i + 1} \subseteq H_{i}$ za vsak $i \ge 1$. 
Rečemo, da se zaporedje $(H_k)_{k \ge 1}$ izteče z grupo $K$, če obstaja naravno število $n$, da velja $H_k = K$ za vsako naravno število $k \ge n$.
\end{definicija}


\begin{definicija}
\label{def_nilpotentna_grupa}
Grupa $G$ je nilpotentna, če se spodnja centralna vrsta $(\gamma_k(G))_{k \ge 1}$, podana rekurzivno z \begin{equation*}
\gamma_1(G) := G \text{ in } \gamma_{k +1}(G) := [\gamma_k(G), G],
\end{equation*}  
izteče s trivialno grupo. Najmanjšemu številu $d$, za katero je $G^{(d)} = \mathbf{1}$ rečemo razred rešljivosti grupe $G$.    
\end{definicija}

Celo družino primerov nilpotentnih grup nam podaja naslednja ugotovitev.

\begin{trditev}
\label{trd_p_grupe_so_nilpotentne}
    Vse $p$-grupe so nilpotentne. Natančneje, če je $\lvert G \rvert  = p^{d}$ za neko naravno število $d \ge 1$, potem je $G$ nilpotentna razreda največ $d$. 
\end{trditev}
\begin{dokaz}
    TODO, tole ne bi smelo biti težko, samo moraš paziti, da je v skladu s spodnjo centralno vrsto.
\end{dokaz}

\begin{primer}
Trditev \ref{trd_p_grupe_so_nilpotentne} nam sporoča, da so vse diedrske grupe oblike $D_{2 \cdot 2^{k}}$ nilpotentne. Izkaže se, da so to tudi vse, saj (TODO tukaj moraš končati razmislek z dokazom https://math.stackexchange.com/questions/834966/is-the-dihedral-group-d-n-nilpotent-solvable). 
\end{primer}

\begin{definicija}
    \label{def_resljiva_grupa}
    Grupa $G$ je rešljiva, če se izpeljana vrsta $(G^{(k)})_{k \ge 0}$, podana rekurzivno z \begin{equation*}
        G^{(0)} := G \text{ in } G^{(k + 1)} := [G^{(k)}, G^{(k)}],
        \end{equation*}  
        izteče s trivialno grupo. Najmanjšemu številu $d$, za katero je $G^{(d)} = \mathbf{1}$ rečemo razred rešljivosti grupe $G$.    
    \end{definicija}
    
    \begin{primer}
    Diedrske grupe $D_{2n}$ so rešljive razreda $2$, saj imamo zaporedje (TODO dokaži, kako izgleda to zaporedje. ).  % https://math.stackexchange.com/questions/834966/is-the-dihedral-group-d-n-nilpotent-solvable
    \end{primer}
    
    \begin{primer}
        Vse nilpotentne grupe so rešljive, saj po definiciji za vsako število $k \ge 0$ velja \begin{equation*}
        G^{(k)} = [G^{(k-1)}, G^{(k-1)}] \subseteq  [\gamma_k(G), G] = \gamma_{k +1}(G).
        \end{equation*}
        Niso pa vse nilpotentne grupe rešljive, primer so recimo diedrske grupe $D_{2n}$, kjer $2n$ ni dvojiška potenca. 
    \end{primer}
    
    % TODO: Naštevalna okolja enumerate, itemize ... uporabljamo za nizanje kratkih izjav, npr. delov trditev, niso pa primerna za pisanje daljših delov besedila. Tudi če v trditvi uporabimo naštevalno okolje, tega ni primerno uporabiti v dokazu te trditve, razen če so dokazi posameznih točk enovrstični. Namesto tega v dokazu točke ločimo tako, da dokaz vsake tvori svoj odstavek, oznako točke pa postavimo na začetek tega odstavka. Če so dokazi posameznih točk sestavljeni iz več odstavkov, lahko dokaze različnih točk bolj nazorno ločimo s prazno vrstico med odstavkoma, ki pripadata različnima točkama, ali pa oznako točke na začetku odstavka napišemo krepko. Podobno velja tudi za druga naštevanja, kjer ob vsaki točki napišemo daljšo razlago.

    \begin{trditev}
    \label{trd_lastnosti_resljivih_grup}
    Za rešljive grupe veljajo naslednje osnovne lasnosti.
     \begin{enumerate}
        \item Vsaka podgrupa rešljive grupe je rešljiva.
        \item Vsak kvocient rešljive grupe je rešljiv.
        \item Naj bo $N \triangleleft G$ in naj bosta $N$ in $G / N$ rešljivi grupi razreda $d_{N}$ oziroma $d_{G / N}$. Potem je $G$ rešljiva grupa razreda največ $d_N + d_{G / N}$.
        \item Naj bosta $M, N \triangleleft G$ rešljivi razreda $d_M$ oziroma $d_N$. Potem je edinka $MN$ rešljiva razreda največ $d_M + d_N$.   
     \end{enumerate}
    \end{trditev}
    \begin{dokaz}
        \begin{enumerate}
            \item To je očitna posledica dejstva, da za $H \le G$ velja $H^{(k)} \subseteq G^{(k)}$ za vsak $k \in \mathbb{N} \cup \left\{ 0\right\}$.
            \item Naj bo $G$ rešljiva in naj bo $N \triangleleft G$. Zaradi rešljivosti grupe $G$ obstaja naravno število $d$, da je $G^{k} \subseteq N$ za vse $k \ge d$, kar implicira $ (G / N)^{(k)} = \left\{ 1_{G / N}\right\}$ za vse $k \ge d$. 
            \item Ker je $G / N$ rešljiva grupa razreda $n_{G / N}$, bo $G^{(k)} \subseteq N$ za vse $k \ge d_{G / N}$. Ker je $N$ rešljiva razreda $d_N$, bo nadalje veljalo $G^{(k)} = \left\{ 1_G\right\}$ za vse $k \ge d_M + d_N$.
            \item Dokaz je prirejen po opombi 4 iz \cite[str.4]{Schneider_2016}. Po drugem izreku o izomorfizmu lahko zapišemo kratko eksaktno zaporedje \begin{equation*}
            \mathbf{1} \to M \to MN \to MN / M \cong N / (N \cap M) \to \mathbf{1}.
            \end{equation*}  
            Ker je $N$ rešljiva, je po drugi točki trditve  njen kvocient $ N / (N \cap M)$ rešljiv razreda največ $d_N$ in posledično tudi kvocient $ MN / M $. Ker je $M$ rešljiva razreda $d_M$, po tretji točki trditve sledi $(MN)^{(k)} = \left\{ 1_{G}\right\}$ za vse $k \ge d_N + d_M$.
        \end{enumerate}
    \end{dokaz}

    Razširitvena lema nam ponuja naslednjo skromno oceno dolžine kratkih netrivialnih zakonov v rešljivih oziroma nilpotentnih grupah.

    \begin{trditev}
    \label{trd_ocitna_meja_za_kratke_zakone_resljive_grupe}
     Obstaja beseda $w \in F_2$ dolžine $l(w) \le 4^{d}$, ki je zakon v vseh grupah razreda rešljivosti (ali nilpotentnosti) $d$ ali manj.  
    \end{trditev}
    \begin{dokaz}
        Trditev je posledica razširitvene leme \ref{lem_razsiritvena_lema}, dokaz poteka z indukcijo po $d$. Za $d = 1$ je grupa $G$ Abelova, zato je ustrezni zakon beseda $w = [x,y]$, ki je dolžine 4.
        Za $d > 1$ opazimo, da je kvocient $G / G^{(1)}$ Abelova grupa, $G^{(1)}$ pa rešljiva grupa razreda največ $d - 1$. Zato z uporabo razširitvene leme in indukcijske predpostavke najdemo besedo $w \in  F_2$ dolžine \begin{equation*}
            l(w) \le  4 \cdot 4^{d - 1} = 4^{d},
        \end{equation*}  
        ki je zakon za grupo $G$. Za nilpotentne grupe upoštevamo dejstvo $G^{(1)} \subseteq \gamma_1(G)$, kar implicira komutativnost grupe $G / \gamma_1(G)$. 
    \end{dokaz}
    
    \begin{opomba}
    V članku \cite[str.~8]{Kozma_Thom_2016} je podana nekoliko šibkejša meja $l(w) \le  4 \cdot 6^{d-1}$, ker je avtor uporabil šibkejšo obliko razširitvene leme.
    \end{opomba}

\subsection{Konstrukcija kratkih zakonov za nilpotentne in rešljive grupe}

Konstrukcija kratkih zakonov za nilpotentne grupe je opisana v članku \cite{Elkasapy_Thom_2013} in z razlagami dopolnjena v magistrskem delu \cite{Schneider_2016}.
Glavna ideja je, da poiščemo kratke netrivialne besede, vsebovane v izpeljani vrsti proste grupe $F_2 = \langle a, b \rangle $. Najprej definiramo zaporedji $(a_n)_n$ in $(b_n)_n$ v $F_2$ s predpisoma
\begin{equation*}
a_0 = a, \, a_{n + 1} = [b_n^{-1}, a_{n}] \text{ in } b_0 = b, \, b_{n + 1} = [a_{n}, b_{n}]. 
\end{equation*}  
Besede, ki jih bomo konstruirali s tema zaporedjema, morajo biti netrivialne, zato potrebujemo naslednjo lemo (lema 3.1 v viru \cite{Kozma_Thom_2016} oziroma lema 8 v \cite{Schneider_2016}).
\begin{lema}
\label{lem_ni_krajsanj_produkti_ab}
Za vsak $n \in  \mathbb{N}$ so besede $a_{n} a_{n}$, $a_{n}^{-1} a_{n}^{-1}$, $b_{n} b_{n}$, $b_{n}^{-1} b_{n}^{-1}$, $a_{n}^{-1} b_{n}$, $b_{n}^{-1} a_{n}$, $a_{n} b_{n}^{-1}$, $b_{n} a_{n}^{-1}$, $a_{n}^{-1} b_{n}^{-1}$ in $b_{n} a_{n}$ v okrajšani obliki.   
\end{lema}
\begin{dokaz}
    Dokaz poteka z indukcijo po $n$. Za $n = 0$ je tridtev očitna, ker sta $a$ in $b$ različna generatorja grupe $F_2$. Za $n > 0$ razpišimo produkt $a_n a_n$.
    \begin{equation*}
    a_{n} a_{n} = [b_{n- 1}^{-1}, a_{n-1}]^2 = b_{n- 1}^{-1} a_{n-1} b_{n- 1} \underbrace{a_{n-1}^{-1} b_{n- 1}^{-1}}_{\text{ni krajšanja}}  a_{n-1} b_{n- 1} a_{n-1}^{-1} 
    \end{equation*}  
    Ker po indukcijski predpostavki vemo, da ne more priti do krajšanja v produktu $a_{n -1}^{-1} b_{n -1}^{-1}$, ne more priti do krajšanja v produktu $a_{n} a_{n}$ ali njegovem inverzu $a_{n}^{-1} a_{n}^{-1}$. Enako sklepamo za preostale produkte.
    \begin{itemize}
        \item Produkt $b_{n} b_{n}$ in njegov inverz sta okrajšana, ker je okrajšan $b_{n - 1}^{-1} a_{n-1}.$
        \item Produkt $a_{n}^{-1} b_{n}$ in njegov inverz sta okrajšana, ker je okrajšan $b_{n - 1} a_{n-1}.$
        \item Produkt $b_{n} b_{n}^{-1}$ in njegov inverz sta okrajšana, ker je okrajšan $a_{n - 1}^{-1} b_{n-1}.$
        \item Produkt $a_{n}^{-1} b_{n}^{-1}$ in njegov inverz sta okrajšana, ker je okrajšan $b_{n - 1} b_{n-1}.$
    \end{itemize}     
\end{dokaz}

\begin{opomba}
Produkti oblike $a_n b_n$ oziroma njihovi inverzi $b_n ^{-1} a_n ^{-1}$ niso nujno okrajšane besede, na primer že za $n = 1$ dobimo $a_1 b_1 = b^{-1} a b a a^{-1} b a^{-1} b^{-1}$. To dejstvo bomo izkoristili v nadaljevanju. 
\end{opomba}

Najprej se prepričajmo, da so besede $a_n$ oziroma $b_n$ res elementi izpeljane grupe $F_2^{(n)}$.

\begin{lema}
\label{lem_besede_ab_so_elementi_izpeljane_grupe}
\end{lema}
\begin{dokaz}
    Dokaz poteka z indukcijo po $n$. Za $n = 0$ je očitno $a_0 = a \in F_2 = F_2^{(0)}$ in $b_0 = b \in F_2 = F_2^{(0)}$. Za $n > 0$ velja $a_{n + 1} = [b_n^{-1}, a_n] \in \left[ F_2^{(n)}, F_2^{(n)} \right] = F_2^{(n + 1)}$ in $b_{n+1} = [a_{n}, b_{n}] \in  \left[ F_2^{(n)}, F_2^{(n)} \right] = F_2^{(n + 1)}$. 
\end{dokaz}
% TODO tole indukcijo bolje opiši

Nato ocenimo dolžino členov zaporedij $(a_{n})_n$ in $(b_{n})_n$.

\begin{lema}
\label{lem_ocena_dolzine_clenov_zaporedij_ab}
Za vsak $n \in \mathbb{N} \cup \left\{ 0\right\}$ velja $4^{n} \ge l(a_{n}) = l(b_{n}) \ge 2^{n}$.
\end{lema}
\begin{dokaz}
Po definiciji zaporedja $(b_n)_n$ velja \begin{align*}
    l(b_{n+1}) &= l(a_{n} b_{n} a_{n}^{-1} b_{n}^{-1}) \\
     &= l(a_{n} b_{n}) + l(a_{n}) + l(b_{n}) \\
     &= l(b_{n}^{-1} a_{n} b_{n} a_{n}^{-1}) \\
     &= l(a_{n + 1})
\end{align*}
Za sklep v drugi in tretji vrstici je bila potrebna lema \ref{lem_ni_krajsanj_produkti_ab} ter preprost sklep, da za besedi $w_1, w_2 \in F_2$, za kateri je produkt $w_1 w_2$ okrajšan, velja $l(w_1 w_2) = l(w_1) + l(w_2)$.
Iz druge vrstice sledi $l(b_{n+1}) \ge  2 l(b_n)$ od koder z indukcijo dobimo $l(a_{n}) = l(b_{n}) \ge 2^{n}$. Iz tretje vrstice dobimo $l(b_{n+1}) \le 4 l(b_n)$ od koder z indukcijo sledi $l(b_n) \le 4^{n}$, kar zopet dokaže oceno \ref{trd_osnovna_ocena_resljive_grupe}.    
\end{dokaz}
% TODO tole se da neoliko lepše utemeljiti, samo upoštevati je treba trikotniško neenakost za l

Vrednost splošnega člena zaporedja $c_n := l(a_n) = l(b_n)$ nam podaja dolžino netrivialne besede v grupi $F_2^{(n)}$.
\begin{lema}
\label{lem_vrednost_cn}
Zaporedje $(c_n)_n$ ustreza rekurzivni zvezi $c_{n+2} = 3 c_{n+1} + 2c_{n}$ z začetnima členoma $c_0 = 1$ in $c_1 = 4$. Od tod lahko izrazimo \begin{equation*}
c_{n} = \left( \frac{1}{2} + \frac{5}{2 \sqrt{17}} \right) \left( \frac{3 + \sqrt{17} }{2} \right)^{n} +  \left( \frac{1}{2} - \frac{5}{2 \sqrt{17}} \right) \left( \frac{3 - \sqrt{17} }{2} \right)^{n} \le C_1 \iota^{n} + o(1),
\end{equation*}
kjer je $\iota := (3 + \sqrt{17} ) / 2 = 3{,}5615528 \ldots $ in $C_1 = 1/ 2 + 5 / (2 \sqrt{17} )$. 
\end{lema}  
\begin{dokaz}
    Dokaz je v isti obliki podan v \cite{Schneider_2016} in uporablja lemo \ref{lem_ni_krajsanj_produkti_ab}.
    \begin{align*}
        c_{n+2} &= l(b_{n+2}) \\
         &= l([a_{n + 1}, b_{n+1}])\\
         &= l([[b_n ^{-1}, a_{n}], [a_{n}, b_{n}]]) \\
         &= l(b_{n} ^{-1} a_{n} b_{n} \underbrace{a_{n} ^{-1} a_{n}}_{\text{se pokrajša}}  b_{n} a_{n} ^{-1} b_{n} ^{-1}) + l([a_{n}, b_{n} ^{-1}]) + l([b_{n}, a_{n}]) \\
         &= \underbrace{l(b_{n} ^{-1} a_{n} b_{n})}_{l(a_{n + 1}) - l(a_{n}^{-1}) = c_{n+1} - c_{n}}  + \underbrace{l(b_{n}) + l(a_{n} ^{-1}) + l(b_{n} ^{-1})}_{3 c_{n}}  + \underbrace{l([a_{n} , b_{n}^{-1}])}_{l(a_{n + 1}) = c_{n+1}}  + \underbrace{l([b_{n}, a_{n}])}_{l(b_{n+1}) = c_{n+1}}. 
    \end{align*}
    To nam da za $n \in \mathbb{N} \cup  \left\{ 0 \right\}$ želeno zvezo $c_{n+2} = 3 c_{n+1} + 2c_{n}$ skupaj z začetnima vrednostima $c_0 = 1$ in $c_1 = 4$, kar nam podaja zvezo \begin{equation*}
    \begin{bmatrix}
        c_{n+ 1}\\
        c_{n} 
    \end{bmatrix} = {\underbrace{\begin{bmatrix}
        3 & 2\\
        1 & 0
    \end{bmatrix}}_{A}}^{n}  \begin{bmatrix}
        4 \\
        1 
    \end{bmatrix} = \begin{bmatrix}
        \frac{3 - \sqrt{17} }{2} & \frac{3 + \sqrt{17} }{2}\\
        1 & 1
    \end{bmatrix} \begin{bmatrix}
        \frac{3 - \sqrt{17} }{2} & 0\\
        0 & \frac{3 + \sqrt{17} }{2}
    \end{bmatrix}^{n} 
    \begin{bmatrix}
        - \frac{1}{\sqrt{17} } & \frac{1}{2} + \frac{3}{2 \sqrt{17} }\\
        \frac{1}{\sqrt{17} } & \frac{1}{2} - \frac{3}{2 \sqrt{17} }
    \end{bmatrix}
    \begin{bmatrix}
        4 \\
        1 
    \end{bmatrix}.
    \end{equation*}  
    Z diagonalizacijo matrike $A$ lahko iz druge vrstice razberemo zvezo iz trditve. Neenakost je posledica dejstva,
     da je po absolutni vrednosti največja lastna vrednost matrike $A$ enaka $\iota = (3 + \sqrt{17})  / 2 = 3{,}5615528 \ldots$, kar je asimptotsko gledano veliko boljši rezultat od trditve \ref{trd_ocitna_meja_za_kratke_zakone_resljive_grupe}.
\end{dokaz} 

Direktna posledica te leme je naslednja ugotovitev za rešljive grupe.

\begin{trditev}
\label{trd_osnovna_ocena_resljive_grupe} 
 Obstaja netrivialna besede $w \in F_2$, ki je zakon za vse grupe rešljivostnega razreda $n$ ali manj, dolžine \begin{equation*}
 l(w) \le C_1 \iota^{n} + o(1),
 \end{equation*}  
 kjer sta konstanti $C_1$ in $\iota$ enaki kot v lemi \ref{lem_vrednost_cn}.  
\end{trditev}
\begin{dokaz}
    Naj bo $G$ rešljiva grupa razreda $k$. Za poljubno besedo $w \in F_2^{(n)}$ za vsaka $g, h \in G$ velja -- v skladu z oznakami iz definicije \ref{def_izginjajoca_mnozica} -- $w(g,h) \in G^{(n)}$. 
    Ker je grupa $G$ rešljiva razreda $k \le n$, je $G^{(n)} = G^{(k)} = \left\{  1_G \right\}$,
    torej bo $w$ zakon za grupo $G$. Po prejšnji lemi obstaja netrivialna beseda dolžine $C_1 \iota^{n} +o(1)$ v $F_2^{(n)}$, ki je iskani netrivialni zakon za grupo $G$.
\end{dokaz}

Zdaj moramo to znanje le še prevesti na nilpotentne grupe. Brez dokaza (najdemo ga lahko v \cite[str.~17--18]{Schneider}) bomo privzeli naslednje razmeroma znano dejstvo o členih
spodnje centralne vrste. 

\begin{lema}
\label{lem_povezava_med_spodnjo_in_izpeljano_vrsto}
Za vsak $n \in \mathbb{N} \cup \left\{ 0\right\} $ velja inkluzija
\begin{equation*}
G^{(n)} \subseteq \gamma_{2^{n}}(G).
\end{equation*}    
\end{lema}

Naslednja trditev je kombinacija posledice 4 in leme 11 iz naloge \cite{Schneider_2016}.

\begin{trditev}
\label{trd_koncna_ugotovitev_nilpotentne_v_nalogi}
 Obstaja netrivialna beseda $w \in F_2$, ki je zakon za vse nilpotentne grupe $G$ moči največ $n$, dolžine \begin{equation*}
 l(w) \le  C_3 \log(n)^{\kappa} + o(1),
 \end{equation*}  
 kjer sta $C_3 = 7{,}712869694 \ldots$ in $\kappa = \log_2(\iota) = 1{,}832506 \ldots$ konstanti.    
\end{trditev}

\begin{dokaz}
    Za vsako število $k \in  \mathbb{N}$ je
    število $e = \left\lceil \log_2(k) \right\rceil$ najmanjše naravno število, da velja $k \le 2^{e} \le 2k$.
    Od tod po lemi \ref{lem_povezava_med_spodnjo_in_izpeljano_vrsto} sledi \begin{equation*}
    F_2^{(e)} \subseteq \gamma_{2^{e}}(F_2) \subseteq \gamma_k(F_2).
    \end{equation*}  
     Po trditvi \ref{trd_osnovna_ocena_resljive_grupe} obstaja netrivialna beseda $w \in  F_2^{(e)}$ dolžine največ $C_1 \iota^{e} + o(1)$.
    Zaradi izbira števila $e$ lahko zapišemo \begin{equation*}
    l(w) \le  C_1 \iota^{e} = C_1 2^{\log_2(\iota) e} \le C_1 (2k)^{\log_2(\iota)} = C_1 \iota k^{\log_2(\iota)} = C_2  k^\kappa.
    \end{equation*}
    Nadalje naj bo grupa $G$ nilpotentna razreda $d$, moči $n$ ali manj. Zaradi nilpotentnosti $G$ iz netrivialnosti grupe $\gamma_i(G)$ (za $i \in \mathbb{N} \cup \left\{ 0 \right\}$) sledi netrivialnost kvocienta $\gamma_i(G) / \gamma_{i + 1}(G)$, saj je centralna vrsta pred iztekom strogo padajoča.
    Za razred nilpotentnosti $d$ velja ocena (TODO najdi vir) $d \le \left\lfloor \log_2(G)  \right\rfloor \le \log_2(n)$. Zato po prvem sklepu dokaza obstaja netrivialna beseda $w \in \gamma_d(G)$ dolžine \begin{equation*}
    l(w) \le C_2 d^{\kappa} \le C_2 \log_2(n)^{\kappa} = \frac{C_2}{\log_2(n)^{\kappa}} \log(n)^{\kappa} = C_3 \log(n)^{\kappa}, 
    \end{equation*}  
     saj velja $w \in \gamma_{\left\lfloor \log_2(n) \right\rfloor}(F_2) \subseteq \gamma_{d}(F_2)$. Od tod z analognim razmislekom kot v trditvi \ref{trd_osnovna_ocena_resljive_grupe} sledi, da je $w$ netrivialni zakon za vse nilpotentne grupe razreda $d$. 
\end{dokaz}

V nadaljevanju članka \cite{Elkasapy_Thom_2013} avtorja eksponent $\kappa$ iz prejšnje trditve izboljšata na $\lambda := 1{,}44115577 \ldots$, pri čemer je treba namesto konstante $C_3$ vzeti faktor oblike $8{,}395184144 \ldots + o(1)$. To storita s preučevanjem funkcije \begin{equation*}
\gamma(w) := \max \left\{ n \in  \mathbb{N}  \middle|\, w \in \gamma_{n}(F_2) \right\} \cup \left\{ \infty\right\}. 
\end{equation*}  
Če namreč definiramo $\gamma_n := \gamma(a_{n}) = \gamma(b_{n})$, se da pokazati zvezo $\gamma_{n + 2} - 2 \gamma_{n+1}  - \gamma_{n} \ge  0$ za vse $n \in  \mathbb{N} \cup \left\{ 0\right\}$, s čimer se da po enakem postopku kot v dokazu \ref*{lem_vrednost_cn} izračunati spodnjo mejo $\gamma_n \ge C_4 (1 + \sqrt{2})^{n} - o(1)$.
Avtorja razmislek zaključita z ugotovitvijo, da je namesto eksponenta $\kappa = \log_2( \iota)$ ustrezen $\lambda := \log_{1 + \sqrt{2}}(\iota)$.

Da dobimo primerljiv rezultat za rešljive grupe, se moramo precej bolj potruditi. Postopek je opisan v \cite[str.~3--4]{Thom_2015}, sklicuje se na lastnosti grup avtomorfizmov nilpotentnih grup, ki jih vložimo
v primerne splošne linearne grupe, ki jim lahko dokaj učinkovito ocenimo razred rešljivosti. Ker je jedro te vložitve nilpotentno, se lahko skličemo na izrek \cite{trd_koncna_ugotovitev_nilpotentne_v_nalogi}, kar nam zagotovi naslednji izrek (formulacija iz \cite[str.~25]{Schneider_2016}).  

\begin{izrek}
\label{izr_glavni_izrek_resljive}
 Za vsako število $n \in  \mathbb{N} \cup  \left\{ 0\right\}$ obstaja netrivialna beseda $w \in F_2$ dolžine \begin{equation*}
 l(w) \le (C_{10} + o(1)) \log(n)^{\lambda},
 \end{equation*}  
   ki je zakon za vse rešljive grupe moči $n$ ali manj, kjer sta konstanti enaki $C_{10} := 86.321{,}05422 \ldots$ in $\lambda := 4{,}331612776 \ldots$ 
\end{izrek}

% TODO urediti je treba številčenje konstant v tem poglavju, ni posebej smiselno ...
