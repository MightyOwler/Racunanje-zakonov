\section{Uvod}

% TODO napiši notacijo, recimo za komutator, ciklične grupe, diedrske grupe, SL_2(q), center, spodnjo centralno vrsto, izpeljano vrsto, ...

Abstraktni produkt elementov $a_1, \ldots , a_k$ ter njihovih inverzov $a_1^{-1}, \ldots , a_k^{-1}$, je \emph{$k$-črkovni zakon v grupi $G$}, če ima lastnost, da za vsako zamenjavo $a_1, \ldots, a_k$ s konkretnimi
elementi $g_1, \ldots, g_k \in G$ dobimo rezultat $1_G\in G$. Zakonu $1$ pravimo \emph{trivialni zakon}, ki v kontekstu raziskovanja zakonov ni posebej zanimiv.

Najosnovnejši primer netrivialnega dvočrkovnega zakona se pojavi pri Abelovih grupah. Grupa $G$ je namreč Abelova natanko tedaj, ko za vsaka elementa $g, h \in  G$ velja $gh = hg$, kar je ekvivalentno
zahtevi \begin{equation*}
ghg^{-1}h^{-1} = [g,h] = 1_G.
\end{equation*}

Grupa $G$ je torej Abelova natanko tedaj, ko je štiričrkovna beseda $aba^{-1}b^{-1}$ v njej zakon.

Preučevanje lastnosti zakonov je klasično področje v teoriji grup. V nekem smislu sega že do Abela in Galoisa, saj lahko tako Abelove kot rešljive grupe zelo naravno karakteriziramo s pomočjo zakonov.
Zakone lahko recimo uporabljamo za prezentacijo grup, morda najbolj znani primer predstavljajo diedrske grupe \begin{equation*}
D_{2n} = \langle r, Z  \vert  r^{n} = Z^{2} = 1, (rZ)^2 = 1 \rangle.  
\end{equation*}  
Diedrska grupa je enolično določena s tremi enočrkovnimi zakoni, ki jim morata ustrezati generatorja $r$ in $Z$ ($a^{n}$ in $a^{2}$) oziroma njun produkt $rZ$ ($a^{2}$).
Zakoni so pomembni tudi za obravnavo Bursidovih problemov, ki sprašujejo po končnosti specifičnih kvocientov prostih grup. Gre za klasične probleme, s katerimi so se matematiku ukvarjali že od začetka 20. stoletja.

Glavnina moje diplomske naloge pa se bo v glavnem posvečala nekoliko novejši temi, in sicer preučevanju asimptotskih lastnosti dolžin zakonov, še posebej dvočrkovnih.
Bistvo je še najlažje videti na zgledu simetrične grupe $S_n$. Zanjo po Lagrangeevem izreku velja enočrkovni zakon $a^{n !}$, katerega dolžina znaša $n!$, kar je po Stirlingovi formuli približno.



