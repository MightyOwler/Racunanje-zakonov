\section{Uvod}

% TODO Vse izraze, ki jih uvedemo v definicijah, trditvah ali v veznem besedilu, pišemo poudarjeno, torej znotraj ukaza \emph{...}

% TODO napiši notacijo, recimo za komutator, ciklične grupe, diedrske grupe, SL_2(q), center, spodnjo centralno vrsto, izpeljano vrsto, ...

% TODO tale uvod je treba še kar lepo urediti, da bo v smiselni obliki
% treba je popraviti definicijo zakona, trenutno ni najbolj logično ...
% treba je obrazložiti kaj je sploh poanta diplome

Dvočrkovni zakon v grupi $G$ je abstrakten produkt elementov $x$, $y$ ter njunih inverzov $x^{-1}$ in $y^{-1}$, ki ima lastnost, da za vsako zamenjavo $x$ in $y$ s konkretnima
elementoma $g, h \in G$ dobimo rezultat $1 \in G$.

\begin{opomba}
Definicijo $n$-črkovnih zakonov dobimo tako, da v zgornji definiciji elementa $x$, $y$ (in njuna inverza) nadomestimo z elementi $x_1, x_2, \ldots, x_n$ (in njihovimi inverzi),
ki jih zamenjujemo s konkretnimi elementi $g_1, g_2, \ldots, g_{n} \in G$.
\end{opomba}


Zakonu $1$ pravimo trivialni zakon, ki v kontekstu raziskovanja zakonov ni posebej zanimiv. Najosnovnejši primer netrivialnega zakona se pojavi pri Abelovih grupah, kjer za poljubna elementa $x,y \in  G$ velja $xy = yx$, kar je ekvivalentno
zahtevi \begin{equation*}
xyx^{-1}y^{-1} = [x,y] = 1.
\end{equation*}

Grupa $G$ je torej Abelova natanko tedaj, ko je štiričrkovna beseda $xyx^{-1}y^{-1}$ v njej zakon. 
